\documentclass{article}
\usepackage{algorithm,algorithmic}
\usepackage[utf8]{inputenc}

% Traduction des commandes d'Algorithm
\makeatletter
\renewcommand{\ALG@name}{Algorihtme}
\renewcommand{\algorithmicrequire}{\textbf{Prérequis:}}
\renewcommand{\algorithmicensure}{\textbf{Ensure:}}
\renewcommand{\algorithmicend}{\textbf{fin}}
\renewcommand{\algorithmicif}{\textbf{si}}
\renewcommand{\algorithmicthen}{\textbf{alors}}
\renewcommand{\algorithmicelse}{\textbf{sinon}}
\renewcommand{\algorithmicelsif}{\algorithmicelse\ \algorithmicif}
\renewcommand{\algorithmicendif}{\algorithmicend\ \algorithmicif}
\renewcommand{\algorithmicfor}{\textbf{pour}}
\renewcommand{\algorithmicforall}{\textbf{pour tout}}
\renewcommand{\algorithmicdo}{\textbf{faire}}
\renewcommand{\algorithmicendfor}{\algorithmicend\ \algorithmicfor}
\renewcommand{\algorithmicwhile}{\textbf{tant que}}
\renewcommand{\algorithmicendwhile}{\algorithmicend\ \algorithmicwhile}
\renewcommand{\algorithmicloop}{\textbf{boucle}}
\renewcommand{\algorithmicendloop}{\algorithmicend\ \algorithmicloop}
\renewcommand{\algorithmicrepeat}{\textbf{répéter}}
\renewcommand{\algorithmicuntil}{\textbf{jusqu'à}}
\renewcommand{\algorithmicprint}{\textbf{afficher}}
\renewcommand{\algorithmicreturn}{\textbf{renvoyer}}
\renewcommand{\algorithmictrue}{\textbf{vrai}}
\renewcommand{\algorithmicfalse}{\textbf{faux}}
\renewcommand{\algorithmiccomment}[1]{// #1}


\addtolength{\oddsidemargin}{-.875in}
\addtolength{\evensidemargin}{-.875in}
\addtolength{\textwidth}{1.75in}
\addtolength{\topmargin}{-.875in}
\addtolength{\textheight}{1.75in}

\title{Algorithmes PAF}
\author{Romain Darous, Louis Martinez, Lucas Thomasset, Arthur Toulouse}

\begin{document}

\maketitle

\section{Arbres quaternaires}

Pour implémenter le comportement des boids, ceux-ci ont besoin de connaître leurs voisins.
Cependant cela nécessite à priori d’itérer sur tous les boids présents et ne retenir que ceux qui sont en dessous
d’une distance seuil. Les arbres quaternaires permettent de réduire le nombre d’itérations
pour trouver les voisins d’un boid. (On passe d’une complexité moyenne en O(n²) à une
complexité en O(n.log(n)) )

\medbreak
Un arbre quaternaire \textbf{A} est défini récursivement. Chaque noeud stocke les données suivantes :
\begin{itemize}
    \item Frontière \textbf{f} (rectangulaire) du plan à laquelle il est associé. Elle est définie par le 4-uplet (x, y, w, h) où  (x,y) sont les coordonnées du centre du rectangle, (w,h) sa largeur et sa hauteur.
    \item Référence vers ses 4 fils \textbf{(nw, ne, sw, se)} s’ils sont définis
    \item Liste \textbf{L} des boids qui se trouvent dans la région associée au noeud
    \item Capacité \textbf{c} : nombre maximum d’boids au-delà duquel on définit les 4 fils du noeud (subdivision de la région associée en 4 sous-régions)
\end{itemize}


A chaque actualisation, les n boids (et leur position) sont stockés dans L.
L’arbre A est recréé à chaque fois et on y insère chaque boid.
Il faut itérer deux fois sur l’intégralité des boids : une première fois pour tous les
insérer dans l’arbre et la deuxième pour identifier le voisinnage de chaque boid.


\begin{algorithm}[H]
    \caption{Actualisation à chaque raffraichissement}
    \begin{algorithmic}

        \FORALL{boids b}
        \STATE A.insérer(b)
        \ENDFOR

        \medbreak

        \FORALL{boids b}
        \STATE définir une zone de recherche r centrée sur b
        \STATE A.requête(r,null)
        \ENDFOR
    \end{algorithmic}
\end{algorithm}
\begin{algorithm}[H]
    \caption{insérer(b)}
    \begin{algorithmic}

        \STATE divisé $\leftarrow$ faux
        \IF{f ne contient pas b}
        \RETURN \FALSE
        \ENDIF

        \medbreak

        \IF{L.taille $\leq$ c}
        \STATE L.ajouter(b)
        \RETURN \TRUE
        \ELSIF{A.divisé = faux}
        \STATE subdiviser()
        \ENDIF

        \medbreak

        \RETURN ne.insérer(b) ou nw.insérer(b) ou se.insérer(b) ou sw.insérer(b)
    \end{algorithmic}
\end{algorithm}

\begin{algorithm}[H]
    \caption{subdiviser()}
    \begin{algorithmic}

        \STATE $ne \leftarrow nouvel arbre((x+w/2, y-w/2, w/2, h/2), c)$
        \STATE $nw \leftarrow nouvel arbre((x-w/2, y-w/2, w/2, h/2), c)$
        \STATE $se \leftarrow nouvel arbre((x+w/2, y+w/2, w/2, h/2), c)$
        \STATE $sw \leftarrow nouvel arbre((x-w/2, y+w/2, w/2, h/2), c)$
        \STATE divisé $\leftarrow$ vrai
    \end{algorithmic}
\end{algorithm}


\begin{algorithm}[H]
    \caption{requête(r, T)}
    \begin{algorithmic}

        \IF{T est nulle}
        \STATE initialiser T comme une liste de boids vide
        \ENDIF

        \medbreak

        \IF{f n'intersecte pas r}
        \RETURN T
        \ENDIF

        \medbreak

        \FORALL{boids $\in$ L}
        \STATE ne.requête(r,t)
        \STATE nw.requête(r,t)
        \STATE se.requête(r,t)
        \STATE sw.requête(r,t)
        \ENDFOR

        \medbreak

        \RETURN T
    \end{algorithmic}
\end{algorithm}

\section{Forces subies par les boids}

On suppose que chaque boid connaît ses voisins grâce à la méthode décrite plus haut.
Pour un boid, on définit donc la liste \textbf{V} de ses voisins. Les fonctions suivantes sont définies dans class boid.
\textbf{self} désigne l’instance de boid dans laquelle les fonctions suivantes sont définies.

\begin{algorithm}[H]
    \caption{Force de cohésion}
    \begin{algorithmic}
        \REQUIRE{Liste V des voisins du boid considéré}
        \STATE position\_désirée $\leftarrow$ Vecteur(0,0)
        \STATE force $\leftarrow$ Vecteur(0,0)
        \STATE diff\_angle $\leftarrow$ 0

        \medbreak

        \FORALL{boids b $\in$ V}
        \STATE écart\_position $\leftarrow$ b.position - self.position
        \STATE diff\_angle $\leftarrow$ angle entre position\_désirée et self.vitesse
        \IF{diff\_angle $\leq$ self.angle\_de\_vue}
        \STATE position\_désirée $\leftarrow$ position\_désirée + b.position
        \ENDIF
        \ENDFOR


        \medbreak

        \IF{V.taille $\ge$ 0}
        \STATE position\_désirée $\leftarrow$ position\_désirée / V.taille
        \STATE position\_désirée $\leftarrow$ position\_désirée - self.position
        \STATE position\_désirée.norme $\leftarrow$ self.vitesse\_max
        \STATE force $\leftarrow$ position\_désirée - self.vitesse
        \STATE force.norme\_maximale $\leftarrow$ self.force\_max

        \ENDIF

        \medbreak

        \RETURN force


    \end{algorithmic}
\end{algorithm}

\begin{algorithm}[H]
    \caption{Force de séparation}
    \begin{algorithmic}
        \REQUIRE{Liste V des voisins du boid considéré}
        \STATE position\_désirée $\leftarrow$ Vecteur(0,0)
        \STATE force $\leftarrow$ Vecteur(0,0)
        \STATE diff\_angle $\leftarrow$ 0
        \STATE diff $\leftarrow$ 0

        \medbreak

        \FORALL{boids b $\in$ V}
        \STATE écart\_position $\leftarrow$ b.position - self.position
        \STATE diff\_angle $\leftarrow$ angle entre position\_désirée et self.vitesse
        \IF{diff\_angle $\leq$ self.angle\_de\_vue}
        \STATE diff $\leftarrow$ self.position - b.position
        \STATE diff.normaliser()
        \STATE position\_désirée $\leftarrow$ position\_désirée + diff
        \ENDIF
        \ENDFOR


        \medbreak

        \IF{V.taille $\ge$ 0}
        \STATE position\_désirée $\leftarrow$ position\_désirée / V.taille
        \STATE position\_désirée.norme $\leftarrow$ self.vitesse\_max
        \STATE force $\leftarrow$ position\_désirée - self.vitesse
        \STATE force.norme\_maximale $\leftarrow$ self.force\_max

        \ENDIF

        \medbreak
        \RETURN force


    \end{algorithmic}
\end{algorithm}


\begin{algorithm}[H]
    \caption{Force d'alignement}
    \begin{algorithmic}
        \REQUIRE{Liste V des voisins du boid considéré}
        \STATE vitesse\_désirée $\leftarrow$ Vecteur(0,0)
        \STATE force $\leftarrow$ Vecteur(0,0)
        \STATE diff\_angle $\leftarrow$ 0

        \medbreak

        \FORALL{boids b $\in$ V}
        \STATE écart\_position $\leftarrow$ b.position - self.position
        \STATE diff\_angle $\leftarrow$ angle entre position\_désirée et self.vitesse
        \IF{diff\_angle $\leq$ self.angle\_de\_vue}
        \STATE vitesse\_désirée $\leftarrow$ vitesse\_désirée + b.vitesse
        \ENDIF
        \ENDFOR


        \medbreak

        \IF{V.taille $\ge$ 0}
        \STATE vitesse\_désirée $\leftarrow$ vitesse\_désirée / V.taille
        \STATE vitesse\_désirée.norme $\leftarrow$ self.vitesse\_max
        \STATE force $\leftarrow$ position\_désirée - self.vitesse
        \STATE force.norme\_maximale $\leftarrow$ self.force\_max

        \ENDIF

        \medbreak
        \RETURN force


    \end{algorithmic}
\end{algorithm}

Pour calculer la force d'évitement d'obstacles, le boid cherche la première direction dans laquelle il n'y a plus d'obstacle. Pour cela il balaye son champ de vision sur un nombre déterminé de directions (\textbf{nombre\_points}). Le \textbf{facteur d'anticipation} permet de faire varier la distance à partir de laquelle un boid modifie sa trajectoire.
\begin{algorithm}[H]
    \caption{Force d'évitement d'obstacle}
    \begin{algorithmic}
        \REQUIRE{Liste O des obstacles du terrain, liste V des boids, facteur\_anticipation, nombre\_points}
        \STATE vitesse\_évitement\_désirée $\leftarrow$ Vecteur(0, 0)
        \STATE champ\_vision\_orienté $\leftarrow$ Vecteur(0, 0)
        \STATE force $\leftarrow$ Vecteur(0, 0)
        \STATE obstacles\_voisins $\leftarrow$ 0
        \medbreak
        \STATE \COMMENT{Paramètres angulaires}
        pas $\leftarrow$ 1/nombre\_points
        theta $\leftarrow$ 0

        \medbreak
        \FORALL{obstacle $\in$ O}
        \WHILE{d $\leq$ facteur\_anticipation $\times$ obstacle.rayon}
        \STATE theta $\leftarrow 2 \times \pi \times pas + 1$
        \STATE champ\_vision\_orienté $\leftarrow$ self.position + self.vitesse.rotation(theta).norme(obstacle.rayon)
        \STATE d $\leftarrow$ distance(champe\_vision\_orienté, obstacle.position)
        \IF{i $\leq$ 0}
        \STATE i $\leftarrow$ -i+1
        \ELSE
        \STATE i $\leftarrow$ -i
        \ENDIF
        \ENDWHILE
        \ENDFOR

        \medbreak

        \IF{theta $\neq$ 0}
        \STATE vitesse\_évitement\_désirée $\leftarrow$ vitesse\_évitement\_désirée + self.vitesse.rotation(theta)
        \STATE obstacles\_voisins $\leftarrow$ obstacles\_voisins + 1
        \ENDIF
        \medbreak
        \STATE i $\leftarrow$ 0
        \STATE theta $\leftarrow$ 0
        \medbreak
        \IF{obstacles\_voisins $\geq$ 1}
        \STATE vitesse\_évitement\_désirée $\leftarrow$ vitesse\_évitement\_désirée / obstacles\_voisins
        \STATE vitesse\_évitement\_désirée.norme $\leftarrow$ self.vitesse\_max
        \STATE force $\leftarrow$  vitesse\_évitement\_désirée - self.vitesse
        \STATE force.norme\_maximale $\leftarrow$ self.force\_max
        \ENDIF

        \medbreak

        \RETURN force

    \end{algorithmic}
\end{algorithm}


\section{Détection des groupes}

En entrée, on prend boids la liste de tous les boids (n = nombre de boids) et groups
la liste de tous les groupes. L’idée de l’algorithme est la suivante : A chaque itération de
draw(), il y a une phase d'initialisation où on ne garde qu’un boid par groupe. Ensuite,
on détermine pour chaque boid la liste de ses voisins (seuil de distance et/ou d'angle de
vitesse). Puis on répartit les boids non-leader dans les groupes déjà existants et on en
crée de nouveau si besoin. Enfin, on traite un cas particulier du à la phase d'initiation
(boid avec plusieurs voisins mais seul dans leur groupe).
L’algorithme a une complexité moyenne en O(n²).

\begin{algorithm}[H]
    \caption{Initialisation de groupes}
    \begin{algorithmic}
        \REQUIRE{Liste G des groupes}
        \FORALL{groupe $\in$ G}
        \STATE ne garder que le premier boid (leader) du groupe
        \STATE attribuer à tous les autres boids le groupe 0
        \ENDFOR

    \end{algorithmic}
\end{algorithm}

\begin{algorithm}[H]
    \caption{Détermination des voisins}
    \begin{algorithmic}
        \REQUIRE{Liste V des boids}
        \FORALL{boid b $in$ V}
        \STATE liste des voisins $\leftarrow$ liste vide
        \STATE nombre total de voisins $\leftarrow$ 0
        \FORALL{boid autre $\in$ V}
        \IF{autre est assez loin du bord}
        \IF{distance(b, autre) $\leq$ seuil et b $\neq$ autre et angle(b.vitesse, autre.vitesse) $\leq$ seuil}
        \STATE liste des voisins.ajouter(b)
        \STATE nombre total de voisins $\leftarrow$ nombre total de voisins + 1
        \ENDIF
        \ELSE
        \IF{distance(b, autre) $\leq$ seuil et b $\neq$ 0}
        \STATE liste des voisins.ajouter(b)
        \STATE nombre total de voisins $\leftarrow$ nombre total de voisins + 1
        \ENDIF
        \ENDIF
        \ENDFOR
        \ENDFOR

    \end{algorithmic}
\end{algorithm}

\begin{algorithm}[H]
    \caption{Formation des groupes}
    \begin{algorithmic}
        \REQUIRE{List V des boids, liste G des groupes}
        \STATE id\_group $\leftarrow$ prochain identifiant disponible \COMMENT{max des id dans G + 1}
        \FORALL{boid b $\in$ V}
        \medbreak
        \IF{b.groupe.id = 0}
        \IF{b n'a pas de voisin}
        \STATE Créer un nouveau groupe grp dans G avec l'id id\_group
        \STATE Ajouter b dans grp
        \STATE id\_group $\leftarrow$ id\_group + 1
        \ENDIF
        \medbreak
        \STATE midID $\leftarrow$ 0
        \STATE res $\leftarrow$ [0] \COMMENT{Liste des id des groupes des voisins de b}
        \medbreak
        \FORALL{boid autre $\in$ liste des voisins de b}
        \IF{autre.id n'est pas dans res}
        \STATE Ajouter autre à res
        \ENDIF
        \STATE midID $\leftarrow$ min(res)
        \ENDFOR
        \medbreak
        \IF{midID = 0}
        \STATE On crée un nouveau groupe group dans groups avec l’id id\_group, et on ajoute boid dans ce groupe
        \FORALL{voisin de b}
        \STATE group.ajouter(voisin)
        \ENDFOR
        \ENDIF
        \medbreak
        \IF{midID $\geq$ 1}
        \STATE finalID $\leftarrow$ élément de res privée de 0 qui correspond au group avec le plus de boids
        \STATE On prend le groupe avec l’id finalID (déjà existant), et on y ajoute boid
        \medbreak
        \FORALL{autres éléments de res privé de 0}
        \STATE On enlève le groupe de group
        \STATE ON prend le groupe avec l'id finalID, et on y ajoute chaque membre des groupes ayant un voisin de b
        \ENDFOR
        \ENDIF
        \ENDIF
        \ENDFOR
    \end{algorithmic}
\end{algorithm}

\begin{algorithm}[H]
    \caption{Cas particulier}
    \begin{algorithmic}
        \REQUIRE{Liste G des groupes}
        \FOR{i allant de 0 à G.taille}
        \medbreak
        \IF{G[i] n'a qu'un seul boid b}
        \IF{b a plusieurs voisins}
        \STATE Enlever group de G et ajouter b au groupe du premier voisin qu'on a trouvé
        \ENDIF
        \ENDIF
        \ENDFOR
        \STATE
    \end{algorithmic}
\end{algorithm}



\section{Enveloppe convexe}

\begin{algorithm}[H]
    \caption{Algorithme de Jarvis}
    \begin{algorithmic}
        \REQUIRE{Liste V des boids}
        \STATE q $\leftarrow$ premier boid de la liste
        \medbreak
        \FORALL{boib b $\in$ V et qui n'est pas le premier}
        \IF{b.y $\leq$ q.y}
        \STATE q $\leftarrow$ v
        \ENDIF
        \ENDFOR
        \medbreak
        \STATE u $\leftarrow$ q
        \STATE p $\leftarrow$ q
        \medbreak
        \WHILE{p $\neq$ u}
        \STATE q $\leftarrow$ premier point de V
        \medbreak
        \FORALL{v $\in$ v après q}
        \medbreak
        \IF{pvq tourne dans le sens contraire des aiguilles d'une montre}
        \STATE q $\leftarrow$ v
        \ENDIF
        \ENDFOR
        \medbreak
        \STATE p.suivant $\leftarrow$ q
        \ENDWHILE

    \end{algorithmic}
\end{algorithm}

\section{Prédiction de trajectoire}

Le but est de calculer une estimation de la trajectoire du barycentre d’un groupe de boids.

On dispose d’un objet ayant les attributs suivants :
\begin{itemize}
    \item La position courante du barycentre du groupe \textbf{position\_barycentre\_courante},
          Une liste de positions estimées \textbf{trajectoires\_prédites}, initialisée avec des vecteurs nuls.
          Une liste de vitesses estimées textbf{vitesses\_prédites}, initialisée avec des vecteurs nuls

    \item Trois tableaux de vecteurs : \textbf{position\_pred}, \textbf{vitesse\_prec}, \textbf{accélération\_prec}, de même taille, initialisés avec des vecteurs nuls. Plus l’indice dans le tableau est grand, plus le vecteur associé désigne un pas de temps récent.

    \item Un booléen \textbf{est\_plein} qui permet de vérifier si ces tableaux ont bien été remplis une fois. Vaut False initialement.

    \item Un booléen \textbf{erreur\_trop\_haute} pour décider de quand recalculer la trajectoire en fonction de l’erreur. Vaut False initialement.

    \item Un entier \textbf{erreur\_max} pour quantifier l’erreur tolérée.

    \item Un entier \textbf{nombre\_données\_prédites} qui donne la taille de la fenêtre de prédiction.

    \item Un entier \textbf{positions\_prédites\_passées} qui permet de compter le nombre de données prédites exploitées par le processus. Elle permet de recalculer la trajectoire prédite lorsque tous les pas de temps pour lesquels la position a été prédite sont écoulés. Elle est initialisée à \textbf{nombre\_données\_prédites - 1}

\end{itemize}

V désigne une liste de vecteurs

\begin{algorithm}[H]
    \caption{Calcul du vecteur moyen}
    \begin{algorithmic}
        \REQUIRE{Liste V de vecteurs}
        \STATE vecteur\_moyen $\leftarrow$ Vecteur(0, 0)
        \medbreak
        \FORALL{vecteur v $\in$ V}
        \STATE vecteur\_moyen $\leftarrow$ vecteur\_moyen + v
        \ENDFOR

        \medbreak

        \STATE vecteur\_moyen $\leftarrow$ vecteur\_moyen / V.taille
        \medbreak
        \RETURN vecteur\_moyen

    \end{algorithmic}
\end{algorithm}

\begin{algorithm}[H]
    \caption{Estimation de l'erreur de position : valeur\_erreur()}
    \begin{algorithmic}

        \STATE erreur $\leftarrow$ min(distance(position\_barycentre\_courante, position\_prédite))
        \RETURN erreur

    \end{algorithmic}
\end{algorithm}

L'algorithme permet de mettre à jour les valeurs des positions passées du barycentre du groupe, qui serviront à calculer les vecteurs accélération et vitesse moyen pour appliquer le modèle de la chute libre (voir la présentation du projet pour plus de détails).
\begin{algorithm}[H]
    \caption{Mise à jour des valeurs précédentes : maj\_valeurs\_précédentes()}
    \begin{algorithmic}

        \STATE On translate les valeurs des tableaux position\_prec, vitesse\_prec et acceleration\_prec d'un indice vers la gauche
        \STATE Pour le plus grand indice on prend :
        \begin{itemize}
            \item Pour position\_prec : position\_barycentre\_courante
            \item Pour vitesse\_prec : différence entre la  position au plus grand indice (qui vient d’être calculée)  et celle à l’indice précédent.
            \item Pour acceleration\_prec : différence entre la vitesse au plus grand indice et celle à l’indice précédent
        \end{itemize}

        \STATE Si la fonction est exécutée au moins autant de fois que la taille des trois tableaux ci-dessus, alors est\_plein $\leftarrow$ Vrai


    \end{algorithmic}
\end{algorithm}



\begin{algorithm}[H]
    \caption{Calcul de la trajectoire}
    \begin{algorithmic}

        \STATE maj\_valeurs\_précédentes()
        \medbreak
        \IF{est\_plein}
        \medbreak
        \IF{prositions\_prédites\_passées = nombre\_données\_prédites -1 ou erreur\_trop\_haute}
        \STATE erreur\_trop\_haute $\leftarrow$ Faux
        \STATE positions\_prédites\_passée $\leftarrow$ 0
        \STATE accélération\_moyenne $\leftarrow$ vecteurMoyen(accélérations\_prec)
        \medbreak
        \FOR{i allant de 0 à nombre\_données\_précédentes}
        \STATE vitesses\_prédites[i] $\leftarrow$ vecteurMoyen(vitesse\_prec) + accélération\_moyenne $\times$ i
        \STATE trajectoire\_prédites[i] $\leftarrow$ 0.5 $\times$ i² $\times$ accélération\_moyenne + vecteurMoyen(vitesse\_prec) $\times$ i + position\_barycentre\_courante

        \ENDFOR
        \medbreak
        \ELSE

        \STATE position\_prédites\_passées $\leftarrow$ positions\_prédites\_passées + 1
        \ENDIF
        \medbreak
        \IF{valeur\_erreur() $\geq$ erreur\_max + 1}
        \STATE erreur\_trop\_haute $\leftarrow$ Vrai
        \ENDIF
        \ENDIF

    \end{algorithmic}
\end{algorithm}

\end{document}